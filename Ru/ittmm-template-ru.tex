% This is a LaTeX template
% for preparing documents for ITTMM conference

\documentclass[60x84/16,8pt]{ittmm}

% Убедительная просьба к авторам не редактировать файл definition.tex
\input{definition}

\begin{document}

% Укажите индекс УДК, соответствующий Вашей работе.
\udc{004.4}

\title{Разработка эффективного алгоритма краткосрочного прогнозирования электропотребления с использованием ансамблирования}

\author[1]{П. Г. Любин}
\author[1]{Е. Ю. Щетинин}

\address[1]{Кафедра прикладной информатики и теории вероятностей,\\
  Российский университет дружбы народов,\\
  ул. Миклухо-Маклая, д.6, Москва, Россия, 117198}

\email{\url{lyubin.p@gmail.com}, \url{riviera-molto@mail.ru}}

\begin{abstract}
Краткосрочное прогнозирование потребления электроэнергии является актуальной задачей 
во многих областях в виду специфичности продукта: нельзя накопить и хранить энергию впрок. 
Во-первых, предприятиям-участникам оптового рынка электроэнергии необходимо заранее 
подавать заявки с плановым потреблением, а энергогенерирующим предприятиям необходимо 
планировать мощности. Во-вторых, подобный показатель может использоваться при построение 
других моделей в качестве признака. При этом потребление электрической энергии каким-либо 
объектом (цехом, промышленным предприятием, энергообъединением и т.п.) является временным 
рядом, так как представляет собой мгновенные значения потребляемой мощности замеренные 
в различные моменты времени с определенной периодичностью. В данной работе продемонстрирован 
простой и эффективный метод краткосрочного прогнозирования электропотребления. Метод основывается 
на ансамблировании базовых моделей (RPART - Recursive PARTitioning, CTREE - Conditional Inference 
Trees) и имеет хороший уровень прогнозирования, который сопоставим с более сложными в использовании 
алгоритмами. Ансамблирование представляет собой алгоритм комбинации набора обученных моделей 
с целью повышения точности прогноза, при этом стараясь избежать переобучения. Существует 
несколько методов ансамблирования, которые имеют свои недостатки и преимущества. В данной работе 
мы использовали метод бэггинга (Bagging = Bootstrap aggregating), который помог улучшить 
прогностическую силу отдельных базовых моделей.
\end{abstract}

\keywords{краткосрочное прогнозирование, ансамблирование, RPART, CTREE, случайные деревья, электропотребление, бэггинг}

% https://www.google.ru/search?q=%D0%B0%D0%BD%D1%81%D0%B0%D0%BC%D0%B1%D0%BB%D0%B8%% D0%BE%D1%80%D0%BE%D0%B2%D0%B0%D0%BD%D0%B8%D0%B5&oq=%D0%B0%D0%BD%D1%81%D0%B0%D0%B% C%D0%B1%D0%BB%D0%B8%D0%BE%D1%80%D0%BE%D0%B2%D0%B0%D0%BD%D0%B8%D0%B5&aqs=chrome..% 69i57j0l5.9853j0j7&sourceid=chrome&ie=UTF-8

% https://alexanderdyakonov.wordpress.com/2017/03/10/c%D1%82%D0%B5%D0%BA%D0%B8%D0%% BD%D0%B3-stacking-%D0%B8-%D0%B1%D0%BB%D0%B5%D0%BD%D0%B4%D0%B8%D0%BD%D0%B3-blendi% ng

% http://www.machinelearning.ru/wiki/images/5/56/Guschin2015Stacking.pdf

% \thanks{Рукопись должна содержать УДК, который рекомендуется брать из
%   следующего источника: \url{http://www.mathnet.ru/udc.pdf}.}

\alttitle{Development of an effective algorithm for short-term forecasting of power consumption using ensemble}

\altauthor[1]{P. G. Lyubin}
\altauthor[1]{E. Yu. Shchetinin}

\altaddress[1]{Department of Applied Probability and Informatics\\
Peoples' Friendship University of Russia\\
Miklukho-Maklaya str. 6, Moscow, 117198, Russia}

\begin{altabstract}
Place here short abstract in English (between 150 and 250 words).
\end{altabstract}

\altkeywords{short-term forecast, ensemble, RPART, CTREE, random trees, power consumption, bagging}

\maketitle

\section{Введение}
\label{sec:intro}
К наиболее распространенным методам прогнозирования временных рядов относятся \cite{Tihonov2006}:
\begin{itemize}
    \item прогнозная экстраполяция
    \item экспертные (интуитивные) методы прогнозирования
    \item корреляционный и регрессионный анализы
    \item прогнозирование на базе ARIMA моделей
    \item адаптивные методы прогнозирования
    \item прогнозирование с использованием искусственных нейронных сетей
    \item прогнозирование с использованием гибридных сетей
\end{itemize}

Перечисленные методы могут применяться для прогнозирования электропотребления и обладают присущими им достоинствами и недостатками, подробно описанными в \cite{Tihonov2006}. В современных работах чаще остальных описываются решения данной задачи с применением искусственных нейронных сетей, к недостаткам которых можно отнести сложность настройки и сложность интерпретации. В данной работе мы используем подход, в котором используется ансамбль методов.

На рисунке ниже изображена динамика почасового потребления электроэнергии в России за 3 недели 2017 года: с 13 июня по 3 июля.
\begin{figure}
  \centering
  \includegraphics[width=0.8\linewidth]{Ru/train_dataset.jpeg}
  \caption{Исходные данные}
  \label{fig:data}
\end{figure}

\section{Метод}
\label{sec:methods}
К наиболее распространенным методам ансамблирования относятся \cite{Tihonov2006}:
\begin{itemize}
    \item простое голосование (Simple Voting)
    \item взвешенное голосование (Weighted Voting)
    \item смесь экспертов (Mixture of Experts [19])
    \item бустинг (Boosting)
    \item бэггинг (Bagging = Bootstrap aggregating)
\end{itemize}
В своей работе мы использовали бэггинг, который был предложен Л. Брейманом в 1996
году [4]. Суть метода заключается в формировании различных обучающих подвыборок случайным выбором с возвращениями - некоторые объекты попадают в подвыборку несколько раз, некоторые ни разу. Базовые алгоритмы, обученные по подвыборкам, объединяются в композицию с помощью простого голосования. Достоинствами бэггинга являются: во-первых, возможность использования различных базовых алгоритмов, ошибки которых могут быть взаимно компенсированы при голосовании; во-вторых, некоторые обучающие подвыборки могут не содержать объекты-выбросы и алгоритм, построенный по этим подвыборкам, может оказаться точнее алгоритма, построенного по полной выборке. В данной работе в качестве базовых алгоритмов используются RPART и CTREE. Результат прогнозирования потребления электроэнергии на одни сутки вперед приведен на рисунке ниже.
\begin{figure}
  \centering
  \includegraphics[width=0.8\linewidth]{Ru/prediction.jpeg}
  \caption{Прогноз}
  \label{fig:prediction}
\end{figure}


\section{Заключение}
В данной работе продемонстрирован простой и эффективный метод краткосрочного прогнозирования электропотребления, который может использоваться участниками рынка энергии при планировании генерации и при планировании закупок.

\begin{thebibliography}{99}

\bibitem{Shetinin}
Е.~Ю. Щетинин, Эффективные компьютерные алгоритмы моделирования спотовых цен на электроэнергию. - Научное обозрение, 2016. №22, 237-242 с.

\bibitem{ShetininKaplunovMarkov}
Е.~Ю. Щетинин, С.~В. Каплунов, П.~Н. Марков, Моделирование спотовых цен на электроэнергию с использованием марковских процессов переключения режимов. - Вестник РУДН, Серия Математика. Информатика. Физика, 2012, №3, 61-68 с.

\bibitem{ShetininLyubin}
Е.~Ю. Щетинин, П.~Г. Любин, Робастный алгоритм построения сглаживающих сплайнов. - Научное Обозрение, 2015. №1, 86–94 с.

\bibitem{LyubinShetinin}
П.~Г. Любин, Е.~Ю. Щетинин, Стохастические модели сглаживания и прогнозирования коэффициентов смертности. - Научное Обозрение, 2015, №18, 147–155 с.

\bibitem{Breiman}
L. Breiman, J.~H. Friedman, R.~A. Olshen, C.~J. Stone, Classification and Regression Trees. - Wadsworth, California.

\end{thebibliography}


% % Возможно использовать bibtex.
% \bibliographystyle{elsarticle-num}
% \bibliography{ittmm-template-ru}


\makealttitle      

\end{document}
