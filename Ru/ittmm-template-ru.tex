% This is a LaTeX template
% for preparing documents for ITTMM conference

\documentclass[60x84/16,8pt]{ittmm}

% Убедительная просьба к авторам не редактировать файл definition.tex
\input{definition}

\begin{document}

% Укажите индекс УДК, соответствующий Вашей работе.
\udc{519.25}

\title{Разработка эффективного алгоритма краткосрочного прогнозирования
  электропотребления с использованием метода ансамбля}

\author[1]{Е. Ю. Щетинин}
\author[2]{М. В. Бережков}
\author[2]{П. Г. Любин}

\address[1]{Всероссийский научно-исследовательский институт\\
  по проблемам гражданской обороны  и чрезвычайных ситуаций\\
  МЧС России (федеральный центр науки и высоких технологий)\\
  ул. Давыдковская, д.7, Москва, Россия, 121352}
\address[2]{Федеральное государственное бюджетное образовательное учреждение\\
  Высшего образования\\
  Московский государственный технологический университет ``СТАНКИН''\\
  пер. Вадковский, д.3а, Москва, Россия, 127055}

\email{\url{lyubin.p@gmail.com}, \url{riviera-molto@mail.ru}}

\begin{abstract}
Краткосрочное прогнозирование потребления электроэнергии является актуальной
задачей во многих областях в виду специфичности продукта: нельзя накопить и
хранить энергию впрок. Во-первых, предприятиям-участникам оптового рынка
электроэнергии необходимо заранее подавать заявки с плановым потреблением, а
энергогенерирующим предприятиям необходимо планировать мощности. Во-вторых,
подобный показатель может использоваться при построение других моделей в
качестве признака. При этом потребление электрической энергии каким-либо
объектом (цехом, промышленным предприятием, энергообъединением и т.п.) является
временным рядом, так как представляет собой мгновенные значения потребляемой
мощности замеренные в различные моменты времени с определенной периодичностью. В
данной работе продемонстрирован простой и эффективный метод краткосрочного
прогнозирования электропотребления. Подход основывается на методе ансамбля
базовых моделей (RPART - Recursive PARTitioning, CTREE - Conditional Inference
Trees \cite{BreimanEtAl}) и имеет хороший уровень прогнозирования, который
сопоставим с более сложными в использовании алгоритмами. Метод ансамбля
представляет собой алгоритм комбинации набора обученных моделей с целью
повышения точности прогноза, при этом стараясь избежать переобучения. Существует
несколько методов ансамбля, которые имеют свои недостатки и преимущества. В
данной работе мы использовали метод бэггинга (Bagging - Bootstrap aggregating
\cite{Breiman1996}), который помог улучшить прогностическую силу отдельных
базовых моделей.
\end{abstract}

\keywords{краткосрочное прогнозирование, метод ансамбля, RPART, CTREE, 
  случайные деревья, электропотребление, бэггинг}

% \thanks{Рукопись должна содержать УДК, который рекомендуется брать из
%   следующего источника: \url{http://www.mathnet.ru/udc.pdf}.}

\alttitle{Development of an effective algorithm for short-term forecasting of 
  power consumption using ensemble}

\altauthor[1]{E. Yu. Shchetinin}
\altauthor[2]{M. V. Berezhkov}
\altauthor[2]{P. G. Lyubin}

\altaddress[1]{All-Russian Research Institute\\
  for Civil Defense and Emergencies of the MESRF\\
  (Science and High Technology Federal Center)\\
  Davydkovskaya str., 7, Moscow, 121352, Russia}
\altaddress[2]{Moscow State University of Technology ``STANKIN'',\\ 
  Vadkovsky lane, 3a, Moscow, 127055, Russia}

\begin{altabstract}
Short-term forecasting of electricity consumption is topical task in many areas
in view of the specificity of the product: one can not accumulate and store
energy for future use. First, the wholesale market participants electricity
needs to be submitted in advance with planned consumption, and energy-generating
enterprises need to plan capacity. Secondly, Such an indicator can be used to
construct other models in as a sign. At the same time, the consumption of
electrical energy by some object (shop, industrial enterprise, power
association, etc.) is time series, since it represents the instantaneous values
​​of the power measured at different times with a certain periodicity. AT This
work demonstrates a simple and effective method of short-term forecasting of
power consumption. The approach is based on the ensemble method basic models
(RPART - Recursive PARTitioning, CTREE - Conditional Inference Trees
\cite{BreimanEtAl}) and has a good level of forecasting that is comparable to
more difficult to use algorithms. The ensemble method is an algorithm
combination of a set of trained models in order to improve the accuracy of the
forecast; This is trying to avoid retraining. There are several methods of the
ensemble, which have their own shortcomings and advantages. In this paper we
used Bagging - Bootstrap aggregating method, which helped to improve predictive
strength of individual base models.
\end{altabstract}

\altkeywords{short-term forecast, ensemble, RPART, CTREE, random trees,
  power consumption, bagging}

\maketitle

\section{Введение}
\label{sec:intro}
К наиболее распространенным методам прогнозирования временных рядов относятся:
\begin{itemize}
    \item прогнозная экстраполяция
    \item экспертные (интуитивные) методы прогнозирования
    \item корреляционный и регрессионный анализы
    \item прогнозирование на базе ARIMA моделей
    \item адаптивные методы прогнозирования
    \item прогнозирование с использованием искусственных нейронных сетей
    \item прогнозирование с использованием гибридных сетей
\end{itemize}

Перечисленные методы могут применяться для прогнозирования электропотребления и
обладают присущими им достоинствами и недостатками. В современных работах чаще
остальных описываются решения данной задачи с применением искусственных
нейронных сетей, к недостаткам которых можно отнести сложность настройки и
сложность интерпретации. В данной работе мы используем подход, в котором
используется ансамбль моделей.

На рисунке \ref{fig:data} изображена динамика почасового потребления электроэнергии в
России за 3 недели 2017 года: с 13 июня по 3 июля.
\begin{figure}
  \centering
  \includegraphics[width=0.6\linewidth]{Ru/train_dataset.jpeg}
  \caption{Исходные данные}
  \label{fig:data}
\end{figure}

\section{Метод}
\label{sec:methods}
К наиболее распространенным методам методам ансамбля относятся:
\begin{itemize}
    \item простое голосование (Simple Voting)
    \item взвешенное голосование (Weighted Voting)
    \item смесь экспертов (Mixture of Experts)
    \item бустинг (Boosting)
    \item бэггинг (Bagging - Bootstrap aggregating \cite{Breiman1996})
\end{itemize}
В своей работе мы использовали бэггинг, который был предложен Л. Брейманом в
1996 году. Суть метода заключается в формировании различных обучающих подвыборок
случайным выбором с возвращениями - некоторые объекты попадают в подвыборку
несколько раз, некоторые ни разу. Базовые алгоритмы, обученные по подвыборкам,
объединяются в композицию с помощью простого голосования. Достоинствами бэггинга
являются: во-первых, возможность использования различных базовых алгоритмов,
ошибки которых могут быть взаимно компенсированы при голосовании; во-вторых,
некоторые обучающие подвыборки могут не содержать объекты-выбросы и алгоритм,
построенный по этим подвыборкам, может оказаться точнее алгоритма, построенного
по полной выборке. В данной работе в качестве базовых алгоритмов используются
RPART и CTREE. Результат прогнозирования потребления электроэнергии на одни
сутки вперед (4 июля 2017 года) приведен на рисунке \ref{fig:prediction}.
\begin{figure}
  \centering
  \includegraphics[width=0.6\linewidth]{Ru/prediction.jpg}
  \caption{Прогноз с использованием бэггинга}
  \label{fig:prediction}
\end{figure}


\section{Заключение}
В данной работе продемонстрирован простой и эффективный метод краткосрочного
прогнозирования электропотребления, который может использоваться участниками
рынка энергии при планировании генерации и при планировании закупок.

\begin{thebibliography}{99}

\bibitem{Shetinin}
Е.~Ю. Щетинин, Эффективные компьютерные алгоритмы моделирования спотовых цен на
электроэнергию. - Научное обозрение, 2016. №22, 237-242

\bibitem{ShetininKaplunovMarkov}
Е.~Ю. Щетинин, С.~В. Каплунов, П.~Н. Марков, Моделирование спотовых цен на
электроэнергию с использованием марковских процессов переключения режимов. -
Вестник РУДН, Серия Математика. Информатика. Физика, 2012, №3, 61-68

\bibitem{ShetininLyubin}
Е.~Ю. Щетинин, П.~Г. Любин, Робастный алгоритм построения сглаживающих сплайнов.
- Научное Обозрение, 2015. №1, 86–94

\bibitem{LyubinShetinin}
П.~Г. Любин, Е.~Ю. Щетинин, Стохастические модели сглаживания и прогнозирования
коэффициентов смертности. - Научное Обозрение, 2015, №18, 147–155

\bibitem{BreimanEtAl}
L. Breiman, J.~H. Friedman, R.~A. Olshen, C.~J. Stone, Classification and
Regression Trees. - Wadsworth, California.

\bibitem{Breiman1996}
L. Breiman, Bagging Predictors. - Machine Learning, 1996, 24, 123-140

\end{thebibliography}


% % Возможно использовать bibtex.
% \bibliographystyle{elsarticle-num}
% \bibliography{ittmm-template-ru}


\makealttitle      

\end{document}
